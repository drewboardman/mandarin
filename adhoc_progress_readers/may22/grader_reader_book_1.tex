\documentclass[20pt]{extarticle}
\usepackage[a4paper,margin=1in]{geometry}
\usepackage{xeCJK}
\usepackage{fontspec}
\usepackage{setspace}
\usepackage{fancyhdr}
\usepackage{ulem}

\setCJKmainfont{Noto Sans SC}
\setmainfont{Noto Sans SC}
\linespread{1.8}

\newcommand{\glossedword}[2]{\uline{\textbf{#1}}\textsuperscript{#2}}

\pagestyle{fancy}
\fancyhead{}
\fancyfoot{}
\fancyfoot[C]{\thepage}

\begin{document}

% --- Page 1 ---
\begin{center}
    {\fontsize{36}{44}\selectfont \textbf{间谍的秘密}}
\end{center}
\vspace{2em}
{\fontsize{22}{30}\selectfont
\noindent
\glossedword{王先生}{1}是一个很高的男人。他在一个大\glossedword{公司}{2}工作。\glossedword{王先生}{1}有一个\glossedword{美丽}{3}的妻子,他们有一个儿子。每天,\glossedword{王先生}{1}上班很早,晚上才回家。他说:“工作很重要。”
}
\vfill
{\fontsize{18}{26}\selectfont
\noindent
\textbf{本页生词表:}\\[0.5em]
1. 王先生 (Wáng xiānsheng) — Mr. Wang\\
2. 公司 (gōngsī) — company\\
3. 美丽 (měilì) — beautiful
}
\newpage

% --- Page 2 ---
\begin{center}
    {\fontsize{36}{44}\selectfont \textbf{第二页}}
\end{center}
\vspace{2em}
{\fontsize{22}{30}\selectfont
\noindent
\glossedword{王先生}{1}的朋友\glossedword{李小姐}{4}也在同一个\glossedword{公司}{2}。\glossedword{李小姐}{4}很\glossedword{聪明}{5},也很\glossedword{漂亮}{6}。大家都说\glossedword{王先生}{1}和\glossedword{李小姐}{4}是好朋友,可是\glossedword{王太太}{7}觉得他们有点不一样。
}
\vfill
{\fontsize{18}{26}\selectfont
\noindent
\textbf{本页生词表:}\\[0.5em]
4. 李小姐 (Lǐ xiǎojiě) — Miss Li\\
5. 聪明 (cōngmíng) — smart, clever\\
6. 漂亮 (piàoliang) — pretty, beautiful\\
7. 王太太 (Wáng tàitai) — Mrs. Wang
}
\newpage

% --- Page 3 ---
\begin{center}
    {\fontsize{36}{44}\selectfont \textbf{第三页}}
\end{center}
\vspace{2em}
{\fontsize{22}{30}\selectfont
\noindent
有一天,\glossedword{王太太}{7}看到\glossedword{王先生}{1}的\glossedword{手机}{8}上有\glossedword{李小姐}{4}的名字。她觉得很\glossedword{奇怪}{9},也有点不高兴。\glossedword{王太太}{7}问:“你和\glossedword{李小姐}{4}说什么?”\glossedword{王先生}{1}说:“没什么,是工作。”
}
\vfill
{\fontsize{18}{26}\selectfont
\noindent
\textbf{本页生词表:}\\[0.5em]
8. 手机 (shǒujī) — mobile phone\\
9. 奇怪 (qíguài) — strange
}
\newpage

% --- Page 4 ---
\begin{center}
    {\fontsize{36}{44}\selectfont \textbf{第四页}}
\end{center}
\vspace{2em}
{\fontsize{22}{30}\selectfont
\noindent
晚上,\glossedword{王先生}{1}出去\glossedword{开会}{10}。他说:“今天晚上\glossedword{公司}{2}有事,我要晚一点回家。”可是\glossedword{王太太}{7}觉得不对。她\glossedword{决定}{11}跟着\glossedword{王先生}{1}出去。
}
\vfill
{\fontsize{18}{26}\selectfont
\noindent
\textbf{本页生词表:}\\[0.5em]
10. 开会 (kāihuì) — to have a meeting\\
11. 决定 (juédìng) — decide
}
\newpage

% --- Page 5 ---
\begin{center}
    {\fontsize{36}{44}\selectfont \textbf{第五页}}
\end{center}
\vspace{2em}
{\fontsize{22}{30}\selectfont
\noindent
\glossedword{王太太}{7}看到\glossedword{王先生}{1}走进一家\glossedword{饭店}{12}。\glossedword{李小姐}{4}也在那儿。两个人在说话,看起来很\glossedword{认真}{13}。\glossedword{王太太}{7}没有进去,只是在外面看。
}
\vfill
{\fontsize{18}{26}\selectfont
\noindent
\textbf{本页生词表:}\\[0.5em]
12. 饭店 (fàndiàn) — restaurant\\
13. 认真 (rènzhēn) — serious, earnestly
}
\newpage

% --- Page 6 ---
\begin{center}
    {\fontsize{36}{44}\selectfont \textbf{第六页}}
\end{center}
\vspace{2em}
{\fontsize{22}{30}\selectfont
\noindent
突然,有一个男人走过来。他穿着黑色的\glossedword{衣服}{14},看起来很\glossedword{奇怪}{9}。\glossedword{李小姐}{4}把一个小东西给了这个男人。\glossedword{王先生}{1}看了看\glossedword{四周}{15},很\glossedword{小心}{16}。
}
\vfill
{\fontsize{18}{26}\selectfont
\noindent
\textbf{本页生词表:}\\[0.5em]
14. 衣服 (yīfu) — clothes\\
15. 四周 (sìzhōu) — around\\
16. 小心 (xiǎoxīn) — careful
}
\newpage

% --- Page 7 ---
\begin{center}
    {\fontsize{36}{44}\selectfont \textbf{第七页}}
\end{center}
\vspace{2em}
{\fontsize{22}{30}\selectfont
\noindent
\glossedword{王太太}{7}觉得很不对。她马上打电话给\glossedword{警察}{17}。她说:“这里有\glossedword{奇怪}{9}的人。我觉得他们在做什么\glossedword{坏事}{18}。”
}
\vfill
{\fontsize{18}{26}\selectfont
\noindent
\textbf{本页生词表:}\\[0.5em]
17. 警察 (jǐngchá) — police\\
18. 坏事 (huàishì) — bad thing
}
\newpage

% --- Page 8 ---
\begin{center}
    {\fontsize{36}{44}\selectfont \textbf{第八页}}
\end{center}
\vspace{2em}
{\fontsize{22}{30}\selectfont
\noindent
\glossedword{警察}{17}很快来了。他们问\glossedword{王先生}{1}和\glossedword{李小姐}{4}:“你们在做什么?”\glossedword{王先生}{1}说:“我们在谈工作。”可是\glossedword{警察}{17}在\glossedword{李小姐}{4}的\glossedword{包}{19}里看到一个很小的\glossedword{电脑}{20}。
}
\vfill
{\fontsize{18}{26}\selectfont
\noindent
\textbf{本页生词表:}\\[0.5em]
19. 包 (bāo) — bag\\
20. 电脑 (diànnǎo) — computer
}
\newpage

% --- Page 9 ---
\begin{center}
    {\fontsize{36}{44}\selectfont \textbf{第九页}}
\end{center}
\vspace{2em}
{\fontsize{22}{30}\selectfont
\noindent
\glossedword{李小姐}{4}说:“我不是坏人,我是\glossedword{警察}{17}。我在找一个大坏人。”\glossedword{王先生}{1}也不是坏人,他在\glossedword{帮助}{21}\glossedword{李小姐}{4}。那个穿黑衣服的男人是坏人,他要\glossedword{偷}{22}\glossedword{公司}{2}的\glossedword{秘密}{23}。
}
\vfill
{\fontsize{18}{26}\selectfont
\noindent
\textbf{本页生词表:}\\[0.5em]
21. 帮助 (bāngzhù) — help\\
22. 偷 (tōu) — steal\\
23. 秘密 (mìmì) — secret
}
\newpage

% --- Page 10 ---
\begin{center}
    {\fontsize{36}{44}\selectfont \textbf{第十页}}
\end{center}
\vspace{2em}
{\fontsize{22}{30}\selectfont
\noindent
最后,\glossedword{警察}{17}带走了坏人。\glossedword{王先生}{1}和\glossedword{李小姐}{4}很高兴。\glossedword{王太太}{7}说:“对不起,我不\glossedword{应该}{24}不相信你。”\glossedword{王先生}{1}笑了。他们一家人回家吃饭,很开心。
}
\vfill
{\fontsize{18}{26}\selectfont
\noindent
\textbf{本页生词表:}\\[0.5em]
24. 应该 (yīnggāi) — should
}

\vfill
{\fontsize{20}{28}\selectfont
\noindent
\textbf{结尾:} \\
王先生说:“家和朋友很重要。”大家一起笑了。\\[2em]
\textbf{注:} 本故事尽量只使用你给的词汇表,有极少数高频词(如“警察”等)为情节服务。句子结构简单,适合初学者阅读和复习已学词汇。
}

\end{document}